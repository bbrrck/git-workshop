\documentclass[a4paper,12pt,oneside]{article}
\usepackage{tgpagella}
\usepackage[T1]{fontenc}
\usepackage{parskip}
\usepackage{xcolor}
\usepackage{hyperref}
\hypersetup{
    colorlinks=true,
    % linkcolor=blue,
    % filecolor=magenta,      
    urlcolor=orange,
}

\renewcommand{\contentsname}{Obsah}

\title{Git Workshop - Aj Ty v IT}
\author{Tibor Stanko <\url{tiborstanko.sk}>}
% \date{13.9.2022}

\begin{document}

\maketitle

Tento dokument obsahuje poznámky ku Git workshopu. V prípade otázok ma neváhajte kontaktovať na adrese \url{tibor.stanko@zurich.com}.

\tableofcontents

\section{Čo je to Git?}

Git je systém na kontrolu verzií, v angličtine \textit{version control system} (VCS) alebo \textit{source control management} (SCM).

Git zaznamenáva históriu vývoja projektu. Umožňuje:
\begin{itemize}
      \item vytvárať verzie projektu (\textit{snapshot}y alebo \textit{commit}y);
      \item skákať medzi verziami, vrátiť sa v čase;
      \item vrátiť súbor alebo celý projekt do predošlého stavu;
      \item porovnávať rozdiely medzi rôznymi verziami toho istého súboru;
      \item vytvárať v projekte vetvy (\textit{branches});
      \item zlučovať dve verzie projektu do jednej (\textit{merge});
      \item riešiť konflikty medzi nekompatibilnými verziami (\textit{conflict resolution}).
\end{itemize}

Git sa často používa pri pŕaci v tíme, no je taktiež užitočný aj keď pracujeme sami.

Git nie je len o kóde; dá sa použiť na ukladanie verzií rôznych typov súborov, najmä (no nielen) textových súborov.

\section{Čo je to Github?}

\section{Inštalujeme Git}

\section{Základné Git príkazy}

% \subsection{Terminál}

% \subsection{VS Code}

\section{Pokročilé Git príkazy}

\section{Dôležité koncepty}

\subsection{\textit{Staging area} (Prípravná zóna)}
Niektoré systémy na kontrolu verzií fungujú tak že vytvoria novú verziu zo všetkých aktuálnych súborov v repozitári. Tento spôsob ukladania záloh môže byť nevýhodný. Príkladom je situácia keď sme v repozitári implementovali dve nezávislé funkcie, a chceme ich zachytiť v dvoch rozdielnych verziách. V Gite preto existuje koncept prípravnej zóny (\textit{staging area}), vďaka ktorej máme kontrolu nad tým ktoré zmeny budú a ktoré nebudú pridané do nasledujúcej verzie.

\section{Užitočné zdroje a odkazy}

\begin{itemize}
      \item
            \href{https://git-scm.com}{Oficiálna stránka} a
            \href{https://git-scm.com/doc}{dokumentácia} Gitu
      \item
            \href{https://git-scm.com/book/en/v2}{Pro Git},
            voľne dostupná kniha k dispozícii čiastočne aj
            \href{https://git-scm.com/book/cs/v2}{v češtine}

      \item
            \href{https://www.youtube.com/watch?v=0v5K4GvK4Gs}{Git a Github od základov}:
            videokurz od Yablka

      \item
            Coursera:
            \href{https://www.coursera.org/learn/introduction-git-github/}{Introduction to Git and GitHub}
      \item
            Missing Semester of CS Education,
            \href{https://www.youtube.com/watch?v=2sjqTHE0zok}{Lecture 6: Version Control (git)}

      \item
            \href{https://rogerdudler.github.io/git-guide/}{git - the simple guide}

      \item
            \href{https://medium.com/@praveenmuth2/learn-how-git-works-internally-with-simple-diagrams-a9349dc32831}{Learn how Git works internally with simple diagrams}:
            článok s diagramami na pochopoenie toho ako fungujú Git príkazy

\end{itemize}

\end{document}